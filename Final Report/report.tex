\documentclass[a4paper, 12pt]{article}
\usepackage[T1]{fontenc}
\usepackage[utf8]{inputenc}
\usepackage{lmodern}
\usepackage[english]{babel}
\usepackage{amsmath}
\usepackage{amsfonts}
\usepackage{amssymb}
\usepackage{relsize}
%\usepackage{abraces,mathtools}
%\usepackage[hidelinks,hyperfootnotes=false]{hyperref}
\usepackage{empheq}
\usepackage[coverpage]{polytechnique}

\title{Specificity and constraints in Peptide-Protein bindings in the mouse proteome}
\subtitle{Report for 3rd Year Research Project}
\author{Dhruv SHARMA}
\usepackage{geometry}

\newcommand*\widefbox[1]{\fbox{\hspace{2em}#1\hspace{2em}}}
\begin{document}
\pagebreak
\maketitle
\tableofcontents
\pagebreak
\part{Introduction}
	This report details the work done towards the fulfillment of the requirements of the department of Physics at Ecole Polytechnique. I undertook this research project under the guidance of Dr. Remi Monasson at the Laboratoire de Physique Theoriqu at the Ecole Normale Superieure in Paris. 

	The aim of the project was to study the interactions between short peptide chains and a specific section of signalling proteins called PDZ Domains. One of the aspects that we study here is the specificities of interactions between peptides and PDZ domains. It is well known that macromolecules such as proteins and enzymes interact in a specific manner with other macromolecules and biomolecules. What interested us over the course of the study are the constraints present in the peptide sequences due to the specificity of their interactions with PDZ domains. We will also have an occasion to understand similar constraints on the PDZ domain sequences 

\part{First Model}
\part{Improvements over first model: Bayesian Modeling}
\part{Integrating PDZ Domain sequences} 
\part{Conclusion}

\end{document}
